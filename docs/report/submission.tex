\documentclass[12pt]{spieman}  % 12pt font required by SPIE;
%\documentclass[a4paper,12pt]{spieman}  % use this instead for A4 paper
\usepackage{amsmath,amsfonts,amssymb}
\usepackage{graphicx}
\usepackage{setspace}
\usepackage{tocloft}
\usepackage{hyperref}
\title{Information Retrieval - Submission Report \\Project N.14: Art For Sale}

\author{Jeferson Morales Mariciano}
\author{Filippo Piloni}
\affil{Università della Svizzera italiana, Faculty of Informatics, Lugano, Switzerland}

\renewcommand{\cftdotsep}{\cftnodots}
\cftpagenumbersoff{figure}
\cftpagenumbersoff{table} 
\begin{document}

\maketitle
\tableofcontents
\newpage

\begin{spacing}{1}   % use double spacing for rest of manuscript

    \section{Introduction}
    The Information Retrieval course project \textit{"ArtForSale"} aims to be a working prototype of an information
    retrieval system for a specific task: to display for-sale artworks from art selling related websites
    listed in Section \ref{sec:websites}.
    The system has around $\approx 8,500$ indexed documents served through a
    user-friendly and intuitive interface, further details in UI/UX Section \ref{sec:ui}, \ref{sec:ux}.

    \subsection{Websites}\label{sec:websites}
    The following websites were chosen for the project based on availability of data, ease of scraping and
    low restrictions regarding policies and terms of use to scrape documents i.e. \textit{robots.txt} rules:

    \begin{itemize}
        \item \url{www.artsy.net}
        \item \url{www.artfinder.com}
        \item \url{www.saatchiart.com}
    \end{itemize}

    The total cumulative number of documents gathered from the above websites is $\approx 8,500$ stored in
    \textit{results.json} file containing the list of documents in JSON format.
    Further information concerning the scraping process and implementation is discussed
    in Crawler Section \ref{sec:crawling}.

    \subsection{Features}
    For this project, the following features has been implemented:
    \begin{itemize}
        \item Result presentation
        \item Filtering
        \item Automatic recommendation
    \end{itemize}

    %%%%%%%%%%%%%%%%%%%%%%%%%%%%%%%%%%%%%%%%%%%%%%%%%%%%%%%%%%%%%%%%%
    \section{Technologies Used}

    \subsection{Scrapy}
    Scrapy, a Python framework for web crawling and scraping, is utilized in to initiate the crawling/scraping process for a specified set of artworks. The extracted data is subsequently channeled into a .json file.

    \subsection{PyTerrier}
    PyTerrier is a Python library that provides a high-level interface for the Terrier information retrieval system. In the provided code snippet, PyTerrier is used for building an information retrieval system to search and index documents.\newline
    Our index.py file uses PyTerrier to create the index starting from the .json file

    \subsection{Django \& SQLite}
    The main part of the project has been carried using a Django based backend. The backend handles the main fetches from the frontend.
    The backend is also the joint between the frontend and the index, making calls to a FASTApi that returns relevant documents given a query.\newline
    To return to the frontend the main information about the documents, the backend iterates through a database in SQLite that contains the main information for each artwork

    \subsection{Vue.js \& PrimeVue}
    Finally, to develop the frontend, it has been used Vue.js, using the PrimeVue as UI library. Vue.js allows us to create a reactive and dynamic user interface, while PrimeVue provides us with a set of pre-built components and styles.

    %%%%%%%%%%%%%%%%%%%%%%%%%%%%%%%%%%%%%%%%%%%%%%%%%%%%%%%%%%%%%%%%%
    \section{Crawling}\label{sec:crawling}
    \section{Indexing}\label{sec:indexing}

    \section{Backend}
    The backend relies on a SQLite database to store the main information about the documents, making possible to retrieve the relevant ones following a call to the index.\newline
    Django serves as the foundational framework for constructing the primary structure, bringing with it a multitude of advantages such as rapid development, a robust and secure architecture, built-in administrative features, and a thriving community support ecosystem.\newline
    The backend handles the two main fetches of the application:
    \begin{itemize}
        \item /api/documents/get-documents/"query"/: This endpoint facilitates the retrieval of relevant documents based on a provided query string.
        \item /api/recommendation/get-recommended/: This endpoint is tasked with retrieving recommended artworks. By inputting a set of tags, it conducts a search within the index, returning all pertinent documents associated with the specified criteria.
    \end{itemize}

    %%%%%%%%%%%%%%%%%%%%%%%%%%%%%%%%%%%%%%%%%%%%%%%%%%%%%%%%%%%%%%%%%
    \section{Frontend}
    The frontend is the main part of the application, where all the features are implemented.\newline
    For the interface, it has been chosen a simple layout, consisting of only a search-bar and a search button. Once the user performs the search, the aspect of the page changes, adding new components. We will examine each of them:
    \begin{itemize}
        \item First, under the search-bar will appear a new set of control to filter the obtained results according to four main characteristics: site of origin, minimum price, maximum price and tag.
        \item Second, there will be shown some carousels for the recommended results based on the most frequents tags of the resulted artworks. Essentially, once all the artworks related to the query have been retrieved, the frontend extracts from all these artworks the related tags, it chooses the ones with the higher frequency and, using the /api/recommendation/get-recommended/ fetch, it shows a maximum of 5 results related to each tag in a moving carousel
        \item Finally, the artworks retrieved from the query are displayed in a card-like style, showing an insight of the artwork, the author, the title, the price, a short description and the main tags, with a Visit button that will bring the user to the page of that artwork in the original site
    \end{itemize}

    \subsection{User Interface}\label{sec:ui}

    %%%%%%%%%%%%%%%%%%%%%%%%%%%%%%%%%%%%%%%%%%%%%%%%%%%%%%%%%%%%%%%%%
    \section{User Evaluation}

    \subsection{User Experience}\label{sec:ux}

    %%%%%%%%%%%%%%%%%%%%%%%%%%%%%%%%%%%%%%%%%%%%%%%%%%%%%%%%%%%%%%%%%
    \section{Running information}

    %%%%%%%%%%%%%%%%%%%%%%%%%%%%%%%%%%%%%%%%%%%%%%%%%%%%%%%%%%%%%%%%%
    \section{Conclusions}

\end{spacing}
\end{document}